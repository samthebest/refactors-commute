% !TEX TS-program = pdflatex
% !TEX encoding = UTF-8 Unicode
% This is a simple template for a LaTeX document using the "article" class.
% See "book", "report", "letter" for other types of document.
\documentclass[9pt]{article} % use larger type; default would be 10pt
\usepackage[utf8]{inputenc} % set input encoding (not needed with XeLaTeX)
\usepackage{amssymb}
\usepackage[a4paper, total={7.5in, 9.5in}]{geometry}
%%% Examples of Article customizations
% These packages are optional, depending whether you want the features they provide.
% See the LaTeX Companion or other references for full information.
%%% PAGE DIMENSIONS
\usepackage{geometry} % to change the page dimensions
\geometry{a4paper} % or letterpaper (US) or a5paper or....
% \geometry{margin=2in} % for example, change the margins to 2 inches all round
% \geometry{landscape} % set up the page for landscape
%   read geometry.pdf for detailed page layout information
\usepackage{graphicx} % support the \includegraphics command and options
\usepackage{xcolor}
% \usepackage[parfill]{parskip} % Activate to begin paragraphs with an empty line rather than an indent
\usepackage{mathrsfs}
%%% PACKAGES
\usepackage{booktabs} % for much better looking tables
\usepackage{array} % for better arrays (eg matrices) in maths
\usepackage{paralist} % very flexible & customisable lists (eg. enumerate/itemize, etc.)
\usepackage{verbatim} % adds environment for commenting out blocks of text & for better verbatim
\usepackage{subfig} % make it possible to include more than one captioned figure/table in a single float
% These packages are all incorporated in the memoir class to one degree or another...
\usepackage[greek,english]{babel}
%%% HEADERS & FOOTERS
\usepackage{fancyhdr} % This should be set AFTER setting up the page geometry
\pagestyle{fancy} % options: empty , plain , fancy
\renewcommand{\headrulewidth}{0pt} % customise the layout...
\lhead{}\chead{}\rhead{}
\lfoot{}\cfoot{\thepage}\rfoot{}
%%% SECTION TITLE APPEARANCE
 % (See the fntguide.pdf for font help)
% (This matches ConTeXt defaults)
%%% ToC (table of contents) APPEARANCE
\usepackage[nottoc,notlof,notlot]{tocbibind} % Put the bibliography in the ToC
\usepackage[titles,subfigure]{tocloft} % Alter the style of the Table of Contents
\renewcommand{\cftsecfont}{\rmfamily\mdseries\upshape}
\renewcommand{\cftsecpagefont}{\rmfamily\mdseries\upshape} % No bold!

\newcommand{\text}{\mbox}
\newcommand{\company}{SomeCompany}
% \usepackage[hidelinks]{hyperref}
\usepackage{hyperref}
% \hypersetup{hidelinks}
%%% END Article customizations

%%% The "real" document content comes below...

\title{\bf{Deriving Functional Programming Laws From Categorical Definitions With Scala Examples}}
\author{Samuel Desmond Savage}
%\date{} % Activate to display a given date or no date (if empty),
         % otherwise the current date is printed 

\begin{document}
\maketitle
\tableofcontents

\newpage 

\section{Abstract and Introduction}

This paper first aims to provide absolute unambiguous clarity to ubiquitously used yet often ill-defined vocabulary within the field of Functional Programming.  For any commonly abused Category Theoretic language found in Functional Programming (e.g. Functor, Monad, Product, Coproduct, Kleisli, etc), we will first disambiguate using mathematical definitions, then using the example of the Category of Scala $S$ (which treats Types as $ob(S)$ and (1-param) functions as $hom(C)$) we formally derive corresponding laws that could (in principle) be written as Property Based tests in Scala.

A large point of the paper will be to show that use of Category Theoretic language can have practical application when these laws are known, because they facilitate refactoring and optimising code. When the laws are not known the vocabulary isn't all that useful to be blunt.

We intend on building up the notion of a Commutative Diagram and show that many refactorings or optimisations are the result of following a Commutative Diagram.

\subsection{Dedication}

It's estimated that Alan Turing's creativity in mathematics and engineering, by essentially inventing the computer, to assist in codebreaking in the second world war reduced the length of the war by 2 to 3 years saving 14 to 21 millions lives.  We dedicate this paper to his memory.

\section{Basic Definitions}

\subsection{Notation - Set Theory and Class Theory}

We will refer to \textbf{classes}, which can be thought of as like \textbf{sets} from Set Theory, except that classes may not necessary satisfy a classical axiomatization of Set Theory (e.g. Zermelo–Fraenkel Set Theory).  The reason for this is that a class may consist of, say ``all sets", or ``all sets that are not members of themselves", which are not valid sets according to ZFC.

To keep notation straightforward we may still use Set Theoretic notation, for example $x \in S$, $A \subset B$ and write functions $f: A \rightarrow B$, but this should be read as $x$ is a member of the class $S$, etc.

\subsection{Definition - Category}

A category $C$ consists of the following three mathematical entities

\begin{enumerate}
    \item A class $ob(C)$, whose elements are called \textbf{objects}
    \item A class $hom(C)$, whose elements are called \textbf{morphisms}, or \textbf{maps}, or \textbf{arrows}
    \begin{enumerate}
        \item Every $f \in hom(C)$ has two (not necessarily distinct) corresponding members of $ob(C)$ called the \textbf{source object} and \textbf{target object}
        \item If $a, b \in ob(C)$ are the source and target objects respectively for some $f$, then we write $f: a \rightarrow b$, or say ``$f$ is a morphism from $a$ to $b$"
        \item For any $a, b \in ob(C)$ we write $hom(a, b)$ to be the class (or \textbf{hom-class}) of all morphisms from $a$ to $b$
    \end{enumerate}
    \item A binary operation $\circ: hom(b, c) \times hom(a, b) \rightarrow hom(a, c)$, called \textbf{Composition of Morphisms}, such that, for any $a, b, c, d \in ob(C)$, $f: a \rightarrow b$, $g: b \rightarrow c$, $h: c \rightarrow d$
    \begin{enumerate}
        \item $\circ: hom(b, c) \times hom(a, b) \rightarrow hom(a, c)$
        \item \textbf{Associativity}: $h \circ (g \circ f) = (h \circ g) \circ f$
        \item \textbf{Identity}: For every object $x \in ob(C)$ there exists $1_x: x \rightarrow x$ (sometimes written $Id_x$) called the \textbf{identity morphism for $x$} where $f \circ 1_a = f = 1_b \circ f$
    \end{enumerate}
    \item asdf
\end{enumerate}

When defining a Category it can often be important to explicitly define what is meant when morphisms are equal.

\subsection{Theorem - Identity is Unique}

For any category $C$ and any object $a \in ob(C)$, the identity morphism is unique

\subsubsection{Proof}

Let $I_1$ and $I_2$ satsify the definition of an identity morphism for $a$\newline
\\
\noindent $I_1 = I_2 \circ I_1 = I_1 \circ I_2 = I_2$ by definition
\\
\noindent Therefore $I_1 = I_2$ \blacksquare

\subsection{Example - Scala Category}

Let $S$ be the following

\begin{enumerate}
    \item $ob(S)$ is the class of Scala types
    \item $hom(S)$ is the class of Scala (1-param pure) functions (or methods)
    \item For any type \texttt{A} in $ob(S)$, this has a class of instances, which we will write $A$, and write \texttt{x:A} to mean $x \in A$ or ``x is of type \texttt{A}"
    \item We write $f: A \rightarrow B$ and \texttt{f:A => B} interchangeably without loss of generality
    \item $\circ$ will correspond to regular functional composition, i.e. $g \circ f$ is simply $\texttt{x => g(f(x))}
\end{enumerate}

We define equality on Scala functions as follows: for any types \texttt{A} and \texttt{B} Scala functions \texttt{f:A => B} and \texttt{g:A => B} are equal if and only if for any instance \texttt{x:A}, \texttt{f(x) = g(x)}.  We will henceforth certain typical refactoring operations preserve equality, like inlining and extracting functions.

\subsection{Theorem - The Scala Category is well defined}

$S$ is a well defined Category.

\subsubsection{Proof}

Given types \texttt{A, B, C, D} and functions \texttt{f:A => B}, \texttt{g:B => C}, \texttt{h:C => D}.\newline
\\
\noindent \textbf{Associativity:} we need to show $h \circ (g \circ f) = (h \circ g) \circ f$, which by definition of $=$ means showing:\newline
\noindent For any \texttt{x:A}, \newline
\texttt{(x => h((x => f(g(x)))(x)))(x)} = \texttt{(x => (x => h(g(x)))(f(x)))(x)}\newline
\\
\noindent \texttt{(x => h((x => f(g(x)))(x)))(x)} = \texttt{(x => h((f(g(x)))))(x)} by inlining lambda function \texttt{x => f(g(x))}\newline
= \texttt{(x => h(f(g(x))))(x)}  by removing redundant parens \newline
= \texttt{(x => (x => h(g(x)))(f(x)))(x)} by extracting out the lambda function \texttt{(x => h(g(x)))}\newline
\\
\noindent \textbf{Identity:} For any given type \texttt{A}, \texttt{id:A => A = (x: A) => x} is clearly the identitiy morphism.  Henceforth we will call this \texttt{id[A]}. \newline
\\
\noindent As required \blacksquare

% TODO Get above indentation to work properly

\subsection{Definition - Functor (Covariant)}

Let $C$ and $D$ be categories. A \textbf{Functor} from $C$ to $D$ is a mapping that

\begin{enumerate}
    \item Associates each object $x \in ob(C)$ to an object $F(x) \in ob(D)$
    \item Associates each morphism $f: x \rightarrow y$ in $hom(C)$ to a morphism $F(f): F(x) \rightarrow F(y)$ in $hom(D)$ such that
    \begin{enumerate}
        \item For every $x \in ob(C)$, $F(1_x) = 1_{F(x)}$
        \item For any morphisms $f: x \rightarrow y$ and $g: y \rightarrow z$ in $hom(C)$, $F(g \circ f) = F(g) \circ F(f)$
    \end{enumerate}
\end{enumerate}

I.e. Functors preserve identity and composition.

\subsection{Definition - Functor (Contravariant)}

TODO

\subsection{Example - Scala Parametric Types with \texttt{map}}
\label{Functor Example}

Let \texttt{F} be a type in Scala with one type parameter, and a method \texttt{map}, then define a Functor $F$ as follows:

\begin{enumerate}
    \item Associates each type $A \in ob(S)$ to \texttt{F[A]}
    \item Associates each function $f: A \rightarrow B$ in $hom(S)$ to a morphism $\texttt{map}_F(f): F[A] \rightarrow F[B]$ where for any \texttt{x:F[A]}, $\texttt{map}_F(f)(x) = \texttt{x.map(f)}$
\end{enumerate}

For most native types, like \texttt{List}, \texttt{Option}, etc the induced Functor is indeed a well defined Functor, i.e. satisfies identity and composition preservation.

\subsection{Example - \texttt{List} induces a well defined Functor}

Notation as before \newline
\\
\noindent \textbf{Identity Preservation:} For any \texttt{x:F[A]}, \newline
$\texttt{map}_F(1_A)(x) = \texttt{map}_F(\texttt{id[A]})(x)$ by definition of $1$ \newline
$= \texttt{x.map}(\texttt{id[A]})$ by \hyperref[Functor Example]{definition} \newline
$= \texttt{x}$ an obvious property of lists \newline
$= \texttt{id[F[A]]}(x)$ by definition of \texttt{id} \newline
$= 1_{\texttt{F[A]}}$ by definition of $1$ \newline
\\
\noindent \textbf{Composition:} For any \texttt{x:F[A]}, \newline
$\texttt{map}_F(g \circ f)(x) = \texttt{map}_F(\texttt{x => g(f(x))})(x)$ by definition of $\circ$ \newline
$= \texttt{x.map(x => g(f(x)))}$ by \hyperref[Functor Example]{definition} \newline
$= \texttt{x.map(f).map(g)}$ an obvious property of lists \newline
$= \texttt{map}_F(g)(\texttt{x.map(f)})$ by \hyperref[Functor Example]{definition} \newline
$= \texttt{map}_F(g)(\texttt{map}_F(f)(x))$ by \hyperref[Functor Example]{definition} \newline
$= \texttt{map}_F(g) \circ \texttt{map}_F(f)(x)$ by definition of $\circ$ \newline
Therefore $\texttt{map}_F(g \circ f) = \texttt{map}_F(g) \circ \texttt{map}_F(f)$ \newline
\\
\noindent As required \blacksquare

\subsection{Definition - Functor Laws}

We say a 1-parameterised type \texttt{F} satisfies the Functor Laws provided for any \texttt{x:F[A]}:
\\
\noindent \texttt{x.map(id[A])} \texttt{must\_===} \texttt{x} \newline
\texttt{x.map(f).map(g)} \texttt{must\_===} \texttt{x.map(g compose f)} \newline

\subsection{Example - \texttt{Encoder.contramap} induces a well defined Contravariant Functor}

\end{document}
