% !TEX TS-program = pdflatex
% !TEX encoding = UTF-8 Unicode
% This is a simple template for a LaTeX document using the "article" class.
% See "book", "report", "letter" for other types of document.
\documentclass[9pt]{article} % use larger type; default would be 10pt
\usepackage[utf8]{inputenc} % set input encoding (not needed with XeLaTeX)
\usepackage{amssymb}
\usepackage[a4paper, total={7.5in, 9.5in}]{geometry}
%%% Examples of Article customizations
% These packages are optional, depending whether you want the features they provide.
% See the LaTeX Companion or other references for full information.
%%% PAGE DIMENSIONS
\usepackage{geometry} % to change the page dimensions
\geometry{a4paper} % or letterpaper (US) or a5paper or....
% \geometry{margin=2in} % for example, change the margins to 2 inches all round
% \geometry{landscape} % set up the page for landscape
%   read geometry.pdf for detailed page layout information
\usepackage{graphicx} % support the \includegraphics command and options
\usepackage{xcolor}
% \usepackage[parfill]{parskip} % Activate to begin paragraphs with an empty line rather than an indent
\usepackage{mathrsfs}
%%% PACKAGES
\usepackage{booktabs} % for much better looking tables
\usepackage{array} % for better arrays (eg matrices) in maths
\usepackage{paralist} % very flexible & customisable lists (eg. enumerate/itemize, etc.)
\usepackage{verbatim} % adds environment for commenting out blocks of text & for better verbatim
\usepackage{subfig} % make it possible to include more than one captioned figure/table in a single float
% These packages are all incorporated in the memoir class to one degree or another...
\usepackage[greek,english]{babel}
%%% HEADERS & FOOTERS
\usepackage{fancyhdr} % This should be set AFTER setting up the page geometry
\pagestyle{fancy} % options: empty , plain , fancy
\renewcommand{\headrulewidth}{0pt} % customise the layout...
\lhead{}\chead{}\rhead{}
\lfoot{}\cfoot{\thepage}\rfoot{}
%%% SECTION TITLE APPEARANCE
 % (See the fntguide.pdf for font help)
% (This matches ConTeXt defaults)
%%% ToC (table of contents) APPEARANCE
\usepackage[nottoc,notlof,notlot]{tocbibind} % Put the bibliography in the ToC
\usepackage[titles,subfigure]{tocloft} % Alter the style of the Table of Contents
\renewcommand{\cftsecfont}{\rmfamily\mdseries\upshape}
\renewcommand{\cftsecpagefont}{\rmfamily\mdseries\upshape} % No bold!

\newcommand{\text}{\mbox}
\newcommand{\company}{SomeCompany}
% \usepackage[hidelinks]{hyperref}
\usepackage{hyperref}
% \hypersetup{hidelinks}
%%% END Article customizations

%%% The "real" document content comes below...

\title{\bf{Deriving Functional Programming Laws From Categorical Definitions With Scala Examples}}
\author{Samuel Desmond Savage}
%\date{} % Activate to display a given date or no date (if empty),
         % otherwise the current date is printed 

\begin{document}
\maketitle
\tableofcontents

\newpage 

\section{Abstract and Introduction}

This paper first aims to provide absolute unambiguous clarity to ubiquitously used yet often ill-defined vocabulary within the field of Functional Programming.  For any commonly abused Category Theoretic language found in Functional Programming (e.g. Functor, Monad, Product, Coproduct, Kleisli, etc), we will first disambiguate using mathematical definitions, then using the example of the Category of Scala $S$ (which treats Types as $ob(S)$ and (1-param) functions as $hom(C)$) we formally derive corresponding laws that could (in principle) be written as Property Based tests in Scala.

A large point of the paper will be to show that use of Category Theoretic language can have practical application when these laws are known, because they facilitate refactoring and optimising code. When the laws are not known the vocabulary isn't all that useful to be blunt.

We intend on building up the notion of a Commutative Diagram and show that many refactorings or optimisations are the result of following a Commutative Diagram.

\subsection{Dedication}

It's estimated that Alan Turing's creativity in mathematics and engineering, by essentially inventing the computer, to assist in codebreaking in the second world war reduced the length of the war by 2 to 3 years saving 14 to 21 millions lives.  We dedicate this paper to his memory.

\section{Basic Definitions}

\subsection{Category}

A category $C$ consists of the following three mathematical entities


asf



asdf


\end{document}
